% -----------------------------------------------
% chktex-file 44
\documentclass[../index.tex]{subfiles}

% -----------------------------------------------

\begin{document}

% -----------------------------------------------
\renewcommand{\sectiontitle}{ASCII}
\section{\sectiontitle}
% There were other text encoding standards before ASCII, but ASCII is one that is still used
% sometimes.
% Today, the international standard character set is Unicode, and the standard character encodings
% are the Unicode Transformation Formats.
%
% We'll examine the ASCII standard and its limitations, and then look at how Unicode and UTF
% attempt to ameliorate these issues.

% ---------------------------
\renewcommand{\currenttitle}{Character set vs.\ character encoding}
\begin{frame}{\currenttitle}
% Before we can talk specifically about ASCII or Unicode, we need to define some terms.
% Historically, many of these terms were generally synonymous, but have since become related
% but distinct terms.
%
% A character set denotes a collection of characters that are used in a written language.
% Think of the English alphabet. The upper and lowercase letters A to Z, puntuation symbols,
% and numerical digits would constitute the character set of basic English.
%
% Now, the coded character set is indicates how each character in the character set should be
% mapped to code points.
% Code points are, in our case, numerical values.
%
% So, each character in the character set is mapped to a a numerical value.
% For example, as we'll see in ASCII, the uppercase A is mapped to the code point value of 65.
%
% Finally, the character encoding indiciates how we map those code points, those numerical
% values, into bytes that we can write to memory or storage.
% Naturally, an encoding also specifies how we'll decode the data back into code points and
% characters.
  \begin{itemize}[leftmargin=*]
    \item[] \textbf{Character set} \textendash{} collection of characters in a written language
    \item[] \textbf{Coded character set} \textendash{} function that maps characters to code points
    \item[] \textbf{Code point} \textendash{} a value denoting a character
    \item[] \textbf{Character encoding}
            \textendash{} mapping of code points to bytes\footnotemark{}
  \end{itemize}
  \footnotetext{More precisely, there's the \textbf{character encoding form} and the
                \textbf{character encoding sequence}}
\end{frame}

% ---------------------------
\renewcommand{\currenttitle}{1963 \textendash{} ASCII}
\begin{frame}{\currenttitle}
% The text encoding standard ASCII was first published in 1963. It was based on telegraph code.
% Designed for use of English, it encodes 0 - 9, a - z, A - Z, and punctuation symbols like
% the period and parenthesis. It also has characters such as the line feed and carriage return.
%
% So when we store text according to the ASCII encoding, we convert each character into its
% corresponding numerical value, and then store that value in binary.
  \textbf{ASCII} \textendash{} American Standard Code for Information Interchange \\

  \vspace*{1em}

  Now referred to officially as \textbf{US-ASCII} by the IANA
\end{frame}

% ---------------------------
\begin{frame}{\currenttitle}
% See here a table of all the characters in the ASCII character set.
% This is first page.
%
% In the first column, we see the code point value (in decimal) assigned to each character.
% In the second column, we see the hex equivalent.
% If you remember from two slides ago, this is called the coded character set.
%
% The lowest values are some non-printable characters.
%
% Notice especially the NUL character, which was assigned the value of zero.
% The NUL character is pretty important in a a variety of different domains, such as printing.
% In the C programming language and some data formats, the NUL character indicates the
% termination of a string.
%
% Notice the numerical characters 0 to 9 are assigned values from 48 to 57.
  \vspace*{1em}

  \scriptsize
  % chktex-file 8
% chktex-file 26
% chktex-file 37
% chktex-file 44

\begin{table}
  \begin{tabular}{|c|c|c||c|c|c||c|c|c||c|c|c|}                       \hline
    Dec & Hex &    & Dec & Hex &    & Dec & Hex &   & Dec & Hex &  \\ \hline
    0  & 00 & NUL  & 16 & 10 & DLE  & 32 & 20 &     & 48 & 30 & 0  \\
    1  & 01 & SOH  & 17 & 11 & DC1  & 33 & 21 & !   & 49 & 31 & 1  \\
    2  & 02 & STX  & 18 & 12 & DC2  & 34 & 22 & "   & 50 & 32 & 2  \\
    3  & 03 & ETX  & 19 & 13 & DC3  & 35 & 23 & \#  & 51 & 33 & 3  \\
    4  & 04 & EOT  & 20 & 14 & DC4  & 36 & 24 & \$  & 52 & 34 & 4  \\
    5  & 05 & ENQ  & 21 & 15 & NAK  & 37 & 25 & \%  & 53 & 35 & 5  \\
    6  & 06 & ACK  & 22 & 16 & SYN  & 38 & 26 & \&  & 54 & 36 & 6  \\
    7  & 07 & BEL  & 23 & 17 & ETB  & 39 & 27 & '   & 55 & 37 & 7  \\
    8  & 08 & BS   & 24 & 18 & CAN  & 40 & 28 & (   & 56 & 38 & 8  \\
    9  & 09 & HT   & 25 & 19 & EM   & 41 & 29 & )   & 57 & 39 & 9  \\
    10 & 0A & LF   & 26 & 1A & SUB  & 42 & 2A & *   & 58 & 3A & :  \\
    11 & 0B & VT   & 27 & 1B & ESC  & 43 & 2B & +   & 59 & 3B & ;  \\
    12 & 0C & FF   & 28 & 1C & FS   & 44 & 2C & ,   & 60 & 3C & <  \\
    13 & 0D & CR   & 29 & 1D & GS   & 45 & 2D & -   & 61 & 3D & =  \\
    14 & 0E & SO   & 30 & 1E & RS   & 46 & 2E & .   & 62 & 3E & >  \\
    15 & 0F & SI   & 31 & 1F & US   & 47 & 2F & /   & 63 & 3F & ?  \\ \hline
  \end{tabular}
  \caption{ASCII table, produced with \texttt{ascii | tail -17}}
\end{table}

  \normalsize
\end{frame}

% ---------------------------
\begin{frame}{\currenttitle}
% These are the rest of the characters specified in ASCII character set.
%
% Notice that the alphabeta starts on 65 with uppercase characters.
% Then there are a few symbols, and then the lowercase characters from 97 to 122.
  \vspace*{1em}

  \scriptsize
  % chktex-file 8
% chktex-file 26
% chktex-file 37
% chktex-file 44

\begin{table}
  \begin{tabular}{|c|c|c||c|c|c||c|c|c||c|c|c|}                                                 \hline
    Dec & Hex &  & Dec & Hex &                & Dec & Hex  &  & Dec & Hex  &                 \\ \hline
    64 & 40 & @  & 80 & 50 & P                & 96 & 60  & `  & 112 & 70 & p                 \\
    65 & 41 & A  & 81 & 51 & Q                & 97 & 61  & a  & 113 & 71 & q                 \\
    66 & 42 & B  & 82 & 52 & R                & 98 & 62  & b  & 114 & 72 & r                 \\
    67 & 43 & C  & 83 & 53 & S                & 99 & 63  & c  & 115 & 73 & s                 \\
    68 & 44 & D  & 84 & 54 & T                & 100 & 64 & d  & 116 & 74 & t                 \\
    69 & 45 & E  & 85 & 55 & U                & 101 & 65 & e  & 117 & 75 & u                 \\
    70 & 46 & F  & 86 & 56 & V                & 102 & 66 & f  & 118 & 76 & v                 \\
    71 & 47 & G  & 87 & 57 & W                & 103 & 67 & g  & 119 & 77 & w                 \\
    72 & 48 & H  & 88 & 58 & X                & 104 & 68 & h  & 120 & 78 & x                 \\
    73 & 49 & I  & 89 & 59 & Y                & 105 & 69 & i  & 121 & 79 & y                 \\
    74 & 4A & J  & 90 & 5A & Z                & 106 & 6A & j  & 122 & 7A & z                 \\
    75 & 4B & K  & 91 & 5B & \lbrack{}        & 107 & 6B & k  & 123 & 7B & \{                \\
    76 & 4C & L  & 92 & 5C & \textbackslash{} & 108 & 6C & l  & 124 & 7C & |                 \\
    77 & 4D & M  & 93 & 5D & \rbrack{}        & 109 & 6D & m  & 125 & 7D & \}                \\
    78 & 4E & N  & 94 & 5E & \^{}             & 110 & 6E & n  & 126 & 7E & \textasciitilde{} \\
    79 & 4F & O  & 95 & 5F & \_               & 111 & 6F & o  & 127 & 7F & DEL               \\ \hline
  \end{tabular}
  \caption{ASCII table, cont.}
\end{table}

  \normalsize
\end{frame}

% ---------------------------
\renewcommand{\currenttitle}{7-bit ASCII encoding}
\begin{frame}[fragile]{\currenttitle}
% There are 16 rows, and 8 column groups. That means there are 128 characters to be encoded.
% The maximum value assigned is 127.
%
% 128 = 2^7.
% That means any character in the ASCII character set can be stored in a minimum of 7 bits.
% Any of these character can be stored in less than a single byte.
%
% This is where we get to the character encoding part.
%
% The ASCII standard initially specified a 7-bit encoding to reduce costs in transmitting
% data.
% Below we can see that encoding the word "Hi" takes up 14 bits.
%
% If we refer back to our chart above, we see that the letter H is mapped to the code point 72,
% or, in hex, 48.
% In binary, that's 1001000.
%
% The lowercase i is mapped to the codepoint 105, or 69 in hex.
% In binary, that's 1101001.
%
% The 7-bit ASCII encoding simply stores these translated code points in order.
% In all, "Hi" takes up 14 bits, the first 7 dedicated to H, and the remaining for i.
%
% Like endianness, bits can be ordered starting from least significant bit or most significant
% bit.
% But in modern computers, where the octet or byte is the basic unit of memory indexing,
% we're not really gonna worry about that.
% In the figure below, we've shown it starting from the most significant bit.
  128 characters \textrightarrow{} $2^7$ \textrightarrow{} 7 bits \\

  \vspace*{1.5em}

  \begin{figure}
    \begin{bytefield}[bitwidth=1em]{14}
      \bitheader{0-13} \\ % chktex 8
      \bitboxes{1}{1001000 1101001}
    \end{bytefield}
    \caption{7-bit ASCII encoding for ``Hi''}
  \end{figure}
\end{frame}

% ---------------------------
\renewcommand{\currenttitle}{8-bit ASCII encoding}
\begin{frame}[fragile]{\currenttitle}
% As 8-bit, 16-bit, and 32-bit computers became more widespread, 8-bit encoding derivatives
% developed. These use all 8-bits of a byte instead of 7.
%
% For some 8-bit derivatives, since only 7 bits are needed to store the character, the
% 8th bit is used as a parity bit for error checking. Others set it to 0.
%
% The 8-bit encoding for "Hi" therefore takes up a 16 bits, or 2 bytes.
% The first 7 bits are dedicated to H, then a parity or empty bit.
% Bits 9 to 15 are for i, and the 16th bit is another parity bit.
%
% Because ASCII-encoded characters only take up a single byte at most, we don't need to
% worry about endianness.
% The first byte will be the first character; the second byte will be the second character.
% Remember, endianness is only relevant to multi-byte data.
  8-bit encoding \textrightarrow{} 7 bits + 1 parity or empty bit

  \vspace*{0.5em}

  \begin{figure}
    \begin{bytefield}[bitwidth=1em,bitheight=\widthof{~Parity},
                      boxformatting={\centering\scriptsize\itshape}]{16} % chktex 6
      \bitheader{0,6,7,8,14,15} \\
      \bitbox{7}{Encoded character} & \bitbox{1}{\rotatebox{90}{Parity}} &
      \bitbox{7}{Encoded character} & \bitbox{1}{\rotatebox{90}{Parity}}
    \end{bytefield}
    \caption{8-bit ASCII encoding for two characters}
  \end{figure}

  \begin{figure}
    \begin{bytefield}[bitwidth=1em]{16}
      \bitheader{0-15} \\ % chktex 8
      \bitboxes{1}{1001000{\ } 1101001{\ }}
    \end{bytefield}
    \caption{8-bit ASCII encoding for ``Hi''}
  \end{figure}
\end{frame}

% ---------------------------
\renewcommand{\currenttitle}{Writing an ASCII decoder}
\begin{frame}{\currenttitle}
% Let's actually write an ASCII decoder for the 8-bit encoding.
% This should be pretty trivial.
%
% Our input will be a vector of unsigned 8-bit integers representing the bytes of memory
% we're reading.
% We'll output a string with the decoded characters.
%
% We simply convert each byte value, or the code point, to its corresponding ASCII character,
% then concatenate the resulting characters into a string.
  \lstinputlisting[language=Rust]{\subdir/32-decoder.rs}
\end{frame}

% ---------------------------
\renewcommand{\currenttitle}{What about all the other characters out there?}
\newcommand{\nonasciiitem}[2][???]{\item[] #2 \textrightarrow{} #1}
\begin{frame}{\currenttitle}
% As you can probably guess, there are some pretty obvious caveats to ASCII.
%
% Because the encoding stores one character in one byte, we can't store much else.
% If we make full use of all 8 bits, we can store another 128 characters, pushing our total
% of supported characters up to 256.
%
% But there are obviously more than 256 characters in the world.
% What if we want to write an e with an acute accent to write words like résumé, or write
% French?
% What about Chinese characters or Japanese kanji? How do we encode and store that?
  \begin{itemize}[leftmargin=*]
    \nonasciiitem[101]{e}
    \nonasciiitem{é}
    \nonasciiitem{私}
  \end{itemize}
\end{frame}

% ---------------------------
\renewcommand{\currenttitle}{Extended ASCII character sets}
\begin{frame}{\currenttitle}
% In attempts ameliorate these shortcomings, extended ASCII character sets.
%
% Probably most used is the ISO 8859, more commonly known as ISO Latin 1.
% It extends the ASCII character set by using all 8-bits to store extra characters commonly
% used in writing western European languages.
%
% These we call true extended ASCII, because they preserve the original ASCII character set.
% There are also non-true extended ASCII character sets.
%
% How we'd encode these true extended 8-bit ASCII character sets is the same as with ordinary
% US-ASCII.
  \textbf{ISO/IEC 8859\textendash{1}} \textendash{}
    ASCII + characters sufficient for most western European languages \\
  \textbf{ISO/IEC 8859\textendash{2}} \textendash{}
    like ISO 8859\textendash{}1, but for \textit{eastern} European languges \\

  \vspace*{1em}

  \begin{table}
    \begin{tabular}{|c||c|c|c|c|c|c|c|c|}
      \hline
      char & À  & Á  & Ç  & È  & ä  & æ  & ñ  & ø    \\ \hline
      dec  & 192& 193& 199& 200& 228& 230& 241& 248  \\
      hex  & C0 & C1 & C7 & C8 & E4 & E6 & F1 & F8   \\
      \hline
    \end{tabular}
    \caption{Select characters from ISO 8859\textendash{1}}
  \end{table}
\end{frame}

% ---------------------------
\renewcommand{\currenttitle}{Multi-byte extended ASCII encodings}
\begin{frame}{\currenttitle}
% There are also multi-byte extended ASCII encodings.
%
% Shift JIS is an example of a multi-byte encoding. However, it's technically not true extended
% ASCII because code point 0x24 is mapped to the general currency symbol rather than the dollar
% sign.
%
% Another is UTF-8, and this is the one we'll be looking at next.
  \textbf{Shift JIS} \textendash{}
    Shift Japanese Industrial Standards, for encoding the Japanese language \\
  \textbf{UTF-8} \textendash{}
    Unicode Transformation Set \textendash{} 8-bit
\end{frame}

% -----------------------------------------------

\end{document}
