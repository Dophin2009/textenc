% -----------------------------------------------
\documentclass[../index.tex]{subfiles}

\newcommand\MemoryLayout[1]{
  \begin{tikzpicture}[scale=0.3]
    \draw[thick](0,0)--++(0,3)node[above]{$0$};
    \foreach \pt/\col/\lab [remember=\pt as \tp (initially 0)] in {#1} {
      \foreach \a in {\tp,...,\pt-1} {
        \draw[fill=\col](-\a,0) rectangle ++(-1,2);
      }
      \draw[thick](-\pt,0)--++(0,3)node[above]{$\pt$};
      \if\lab\relax\relax\else
        \draw[thick,decorate, decoration={brace,amplitude=4mm}]
          (-\tp,-0.2)--node[below=4mm]{\lab} (-\pt,-0.2);
      \fi
    }
  \end{tikzpicture}
}

% -----------------------------------------------

\begin{document}

% -----------------------------------------------
\renewcommand{\sectiontitle}{A bit of linguistics and typography}
\section{\sectiontitle}

% ---------------------------
\begin{frame}{\sectiontitle}
  \only<+->{A \textbf{character} is the basic symbol used to write or print a language} \\
  \only<+->{A \textbf{character set} is a collection of characters to write a language} \\
  \only<+->{A \textbf{glyph} is the \textit{visual} representation of a character}
\end{frame}


% -----------------------------------------------

\end{document}
