% -----------------------------------------------
\documentclass[../index.tex]{subfiles}

% -----------------------------------------------

\begin{document}

% -----------------------------------------------
\renewcommand{\sectiontitle}{A memory refresher}
\section{\sectiontitle}
% Before we get into how computers handle text, we need to talk about memory.

% ---------------------------
\renewcommand{\currenttitle}{Bits and bytes}
\begin{frame}{\currenttitle}
% Computers store information bits, a portmanteau of "binary digit", which represent an "on"
% and an "off" state in the abstract sense.
%
% In the physical device, this might be two stable states of a flip-flop, two positions of an
% electrical switch, two distinct voltage or current levels allowed by a circuit, two distinct
% levels of light intensity.
  \only<+->{Computers store information in \textbf{bits} (0 or 1 / \texttt{TRUE} or \texttt{FALSE})} \\
\end{frame}

% ---------------------------
\begin{frame}{\currenttitle}
% We can do binary logic operations on bits, which are the basis for other operations such as
% addition and subtraction.
  \newcommand{\false}{\texttt{0}}
  \newcommand{\true}{\texttt{1}}

  We can do binary logic and arithmetic with these:

  \begin{table}
    \begin{tabular}{c c c c}
      a & b & op & result \\
      \hline{}
      \false{} & \true{} & AND & \false{} \\
      \false{} & \true{} & OR & \true{} \\
      \true{} & \true{} & NAND & \false{} \\
      \false{} & \false{} & NOR & \false{} \\
      \true{} & \true{} & XOR & \false{}
    \end{tabular}
    \caption{Logic binop inputs and results}
  \end{table}
\end{frame}

% ---------------------------
\begin{frame}{\currenttitle}
% Anyway, that's not really relevant to us at the moment.
% As you probably know, a byte is equal to 8 bits.
% This means there are 2^8 possible values contained in a single byte of data.
%
% In modern computers, memory is indexed by bytes. So while the bit is the smallest unit,
% you're typically going to be concerned mostly with bytes.
  A \textbf{byte} is equal to 8 bits \\

  Possible values are in the interval [$0_{10}$, $255_{10}$]
\end{frame}

% ---------------------------
\begin{frame}{\currenttitle}
% Think of some data types that you've used in your code.
% This table has some of the sizes of Java primitives.
%
% An int in Java is a 32-bit integer, so it can store values from -2^31 to 2^31 - 1.
% Because Java doesn't distinguish between unsigned and signed number values like C, Rust,
% or other langauges do, the first bit is dedicated to indicating the sign.
%
% I'm not really going to get into how floating-point number types like float or double.
%
% Booleans, which are just a true/false value, should theoretically take up a single bit,
% but note that JVM stores booleans to take up a byte. That's because memory is usually
% indexed by bytes, increasing speed at the cost of a little bit of storage overhead.
  \begin{table}
    \begin{tabular}{c c c c}
      type & description & bits & bytes \\
      \hline{}
      boolean & true / false flag & 1 & 1\footnotemark \\
      char & 16-bit character & 16 & 2 \\
      short & 16-bit integer & 16 & 2 \\
      int & 32-bit integer & 32 & 4 \\
      long & 64-bit integer & 64 & 8 \\
      float & 32-bit floating-point & 32 & 4 \\
      double & 64-bit floating-point & 64 & 8 \\
    \end{tabular}
    \caption{Sizes of Java data types}
  \end{table}

  \footnotetext{In Java, booleans usually take up a byte for speed}
\end{frame}

% ---------------------------
\renewcommand{\currenttitle}{Memory storage}
\begin{frame}{\currenttitle}
% So how are these data stored in memory?
\end{frame}

% ---------------------------
\renewcommand{\currenttitle}{Big and little endianness}
\begin{frame}{\currenttitle}
% A type like Java's short, which is 32-bits, is going to take up 4 bytes of memory space.
% Because of historical reasons, this can pose problems.
%
% Let's say you have the decimal integer 168496141. In binary, this is a very long value.
% We'll instead refer to it by its hexadecimal form: 0x0A0B0C0D.

  $168496141_{10}$ = 0x0A0B0C0D\textsubscript{h}

\end{frame}

% ---------------------------
\begin{frame}{\currenttitle}
% If you're curious about the etymology of 'endian', it comes from the 1762 satirical novel
% Gulliver's Travels, by Jonathan Swift.
% In one of Gulliver's journeys, he encounters a land of beings named Lilliputians.
% The Lilliputians have divided into two sects, one that holds the belief that the shell of
% a boiled egg should be broken from the big end, and the other from the little end.
% They're called big-endian and little-endian, and computer scientists took those terms.
% Lilliputian has also made it into the English dictionary and has the meaning tiny or trivial.
  \textbf{Endianness} indicates the order in which memory is read
\end{frame}


% -----------------------------------------------

\end{document}
