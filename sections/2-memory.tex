% -----------------------------------------------
\documentclass[../index.tex]{subfiles}

% -----------------------------------------------

\begin{document}

% -----------------------------------------------
\renewcommand{\sectiontitle}{A memory refresher}
\section{\sectiontitle}
% Before we get into how computers handle text, we need to talk about memory.

% ---------------------------
\renewcommand{\currenttitle}{Bits and bytes}
\begin{frame}{\currenttitle}
% Computers store information bits, a portmanteau of "binary digit", which represent an "on"
% and an "off" state in the abstract sense.
%
% In the physical device, this might be two stable states of a flip-flop, two positions of an
% electrical switch, two distinct voltage or current levels allowed by a circuit, two distinct
% levels of light intensity.
  \only<+->{Computers store information in \textbf{bits} (0 or 1 / \texttt{TRUE} or \texttt{FALSE})} \\
\end{frame}

% ---------------------------
\begin{frame}{\currenttitle}
% We can do binary logic operations on bits, which are the basis for other operations such as
% addition and subtraction.
  \newcommand{\false}{\texttt{0}}
  \newcommand{\true}{\texttt{1}}

  We can do binary logic and arithmetic with these:

  \begin{table}
    \begin{tabular}{c c c c}
      a & b & op & result \\
      \hline{}
      \false{} & \true{} & AND & \false{} \\
      \false{} & \true{} & OR & \true{} \\
      \true{} & \true{} & NAND & \false{} \\
      \false{} & \false{} & NOR & \false{} \\
      \true{} & \true{} & XOR & \false{}
    \end{tabular}
    \caption{Logic binop inputs and results}
  \end{table}
\end{frame}

% ---------------------------
\begin{frame}{\currenttitle}
% Anyway, that's not really relevant to us at the moment.
% As you probably know, a byte is equal to 8 bits.
  A \textbf{byte} is equal to 8 bits \\
\end{frame}

% ---------------------------
\begin{frame}{\currenttitle}
% Think of some data types that you've used in your code.
  \begin{table}
    \begin{tabular}{c c c c}
      type & description & bits & bytes \\
      \hline{}
      boolean & true / false flag & 1 & 1\footnotemark \\
      char & 16-bit character & 16 & 2 \\
      short & 16-bit integer & 16 & 2 \\
      int & 32-bit integer & 32 & 4 \\
      long & 64-bit integer & 64 & 8 \\
      float & 32-bit floating-point & 32 & 4 \\
      double & 64-bit floating-point & 64 & 8 \\
    \end{tabular}
    \caption{Sizes of Java data types}
  \end{table}

  \footnotetext{In Java, booleans usually take up a byte for speed}
\end{frame}

% ---------------------------
\begin{frame}{\currenttitle}

\end{frame}


% -----------------------------------------------

\end{document}
