% -----------------------------------------------
% chktex-file 44
\documentclass[../index.tex]{subfiles}

% -----------------------------------------------

\begin{document}

% -----------------------------------------------
\renewcommand{\sectiontitle}{A memory refresher}
\section{\sectiontitle}
% Before we get into how computers handle text, we need to talk about memory.

% ---------------------------
\renewcommand{\currenttitle}{Bits and bytes}
\begin{frame}{\currenttitle}
% Computers store information bits, a portmanteau of "binary digit", which represent an "on"
% and an "off" state in the abstract sense.
%
% In the physical device, this might be two stable states of a flip-flop, two positions of an
% electrical switch, two distinct voltage or current levels allowed by a circuit, two distinct
% levels of light intensity.
  Computers store information in \textbf{bits} (0 or 1 / \texttt{TRUE} or \texttt{FALSE})
  \vspace*{2em}

  \begin{table}
    \centering
    \begin{tabular}{c|c|c|c|c|c|c|c|c|c}
      \hline
      \dots & 1 & 1 & 0 & 1 & 0 & 1 & 1 & 0 & \dots \\
      \hline
    \end{tabular}
    \caption{Memory sketch}
  \end{table}
\end{frame}

% ---------------------------
\begin{frame}{\currenttitle}
% We can do binary logic operations on bits, which are the basis for other operations such as
% addition and subtraction.
  \newcommand{\false}{\texttt{0}}
  \newcommand{\true}{\texttt{1}}

  We can do binary logic and arithmetic with these:

  \begin{table}
    \begin{tabular}{c c c c}
      a       & b       & op    & result  \\ \hline
      \false  & \true   & AND   & \false  \\ % chktex 1
      \false  & \true   & OR    & \true   \\ % chktex 1
      \true   & \true   & NAND  & \false  \\ % chktex 1
      \false  & \false  & NOR   & \true   \\ % chktex 1
      \true   & \true   & XOR   & \false     % chktex 1
    \end{tabular}
    \caption{Logic binop inputs and results}
  \end{table}
\end{frame}

% ---------------------------
\begin{frame}{\currenttitle}
% Anyway, that's not really relevant to us at the moment.
% As you probably know, a byte is equal to 8 bits.
% This means there are 2^8 possible values contained in a single byte of data.
%
% In modern computers, memory is indexed by bytes. So while the bit is the smallest unit,
% you're typically going to be concerned mostly with bytes.
%
% We can, and we will, use hexadecimal instead of binary. If you don't remember, hexadecimal
% is base 16. The digits are 0 - 9, and then A - F representing 10 - 15. Since a byte is 8
% binary digits, we can easily convert it into 2 hex digits by splitting it into 2 groups of
% 4 binary digits and converting to hex.
  A \textbf{byte} is equal to 8 bits \\

  This means possible values are in the interval [$0_{10}$, $255_{10}$] \\

  \begin{table}
    \centering
    \begin{tabular}{c|c|c|c|c|c|c|c}
      \hline
      1 & 0 & 0 & 1 & 1 & 0 & 1 & 0 \\
      \hline
    \end{tabular}
    \caption{A byte representing the value of 154\textsubscript{10}, or \hex{9A}}
  \end{table}

  We can (and will) use 2 hexadecimal (base 16) numbers to represent a single byte
\end{frame}

% ---------------------------
\begin{frame}{\currenttitle}
% Think of some data types that you've used in your code.
% This table has some of the sizes of Java primitives.
%
% An int in Java is a 32-bit integer, so it can store values from -2^31 to 2^31 - 1.
% Because Java doesn't distinguish between unsigned and signed number values like C, Rust,
% or other langauges do, the first bit is dedicated to indicating the sign.
%
% I'm not really going to get into how floating-point number types like float or double.
%
% Booleans, which are just a true/false value, should theoretically take up a single bit,
% but note that JVM stores booleans to take up a byte. That's because memory is usually
% indexed by bytes, increasing speed at the cost of a little bit of storage overhead.
  \begin{table}
    \begin{tabular}{c c c c}
      type & description & bits & bytes \\
      \hline
      boolean & true / false flag & 1 & 1\footnotemark{} \\
      char & 16-bit character & 16 & 2 \\
      short & 16-bit integer & 16 & 2 \\
      int & 32-bit integer & 32 & 4 \\
      long & 64-bit integer & 64 & 8 \\
      float & 32-bit floating-point & 32 & 4 \\
      double & 64-bit floating-point & 64 & 8
    \end{tabular}
    \caption{Sizes of Java data types}
  \end{table}

  \footnotetext{In Java, booleans usually take up a byte for speed}
\end{frame}

% ---------------------------
\renewcommand{\currenttitle}{Big and little endianness}
\begin{frame}{\currenttitle}
% A type like Java's short, which is 32-bits, is going to take up 4 bytes of memory space.
% Because of historical reasons, this can pose problems.
%
% Let's say you have the decimal integer 168496141. In binary, this is a long value.
% We'll instead refer to it by its hexadecimal form: 0A0B0C0D. The largest byte is 0A,
% then 0B, then 0C, and the smallest byte is 0D.
%
% We call 0A the most significant byte. It has the most significant value.
% 0D is called the least significant byte.
  \textbf{MSB} \textrightarrow{} Most Significant Byte \\
  \textbf{LSB} \textrightarrow{} Least Significant Byte

  \vspace*{1em}

  \begin{figure}
    \begin{bytefield}{32}
      \bitboxes{8}{{MSB} {\dots} {\dots} {LSB}}
    \end{bytefield}
    \caption{Most and least significant bytes in a 4-byte word}
  \end{figure}
\end{frame}

% ---------------------------
\begin{frame}{\currenttitle}
% If you have a number like 168-million blah blah blah, which is 0A0B0C0D in hex,
% 0A is the most significant byte, and 0D is the least significant byte.
  \(168496141_{10}\) = \hex{0A0B0C0D}

  \vspace*{1em}

  \begin{figure}
    \begin{bytefield}{32}
      \bitboxes{8}{{0A (MSB)} {0B} {0C} {0D (LSB)}}
    \end{bytefield}
    \caption{Most and least significant bytes in a \hex{0A0B0C0D}}
  \end{figure}
\end{frame}

% ---------------------------
\begin{frame}[fragile]{\currenttitle}
% Different CPU architectures will read data differently.
% Certain architectures, such IBM's z/Architecture and OpenRISC, read data starts from the
% most significant byte to the least significant byte.
% They would read 0A first, then 0B, then 0C, then 0D.
%
% Other architectures do the opposite.
% Your computer, which likely has an AMD64 / Intel x86-64 architecture CPU, reads data from
% the least significant byte to the most significant byte.
% That means it reads 0D, then 0C, then 0B, then 0A.
%
% There are other architectures, like ARM64, which is growing in popularity, that are
% bi-endian.
%
% Of course, this is how it would write to memory as well.
  \(168496141_{10}\) = \hex{0A0B0C0D}

  \vspace{0.5em}

  \newcommand{\intbytefield}[5]{%
    \begin{figure}
      \begin{bytefield}{32}
        \bitheader{0-31} \\ % chktex 8
        \bitboxes{8}{{\colorbox{black!20}{\textbf{#2}}} {#3} {#4} {#5}}%
      \end{bytefield}
      \caption{#1}
    \end{figure}
  }

  \intbytefield{Big endian memory layout}{0A (MSB)}{0B}{0C}{0D}

  \intbytefield{Little endian memory layout}{0D (LSB)}{0C}{0B}{0A}
\end{frame}

% ---------------------------
\begin{frame}{\currenttitle}
% If you're curious about the etymology of 'endian', it comes from the 1762 satirical novel
% Gulliver's Travels, by Jonathan Swift.
% In one of Gulliver's journeys, he encounters a land of beings named Lilliputians.
% The Lilliputians have divided into two sects, one that holds the belief that the shell of
% a boiled egg should be broken from the big end, and the other from the little end.
% They're called big-endian and little-endian, and computer scientists took those terms.
% Lilliputian has also made it into the English dictionary and has the meaning tiny or trivial.
  \begin{itemize}[leftmargin=*]
    \item[] \textbf{Endianness} \textendash{} the order in which memory is written and read \\
    \item[] \textbf{Big endianness} \textendash{} most significant byte to least significant byte \\
    \item[] \textbf{Little endianness} \textendash{} least significant byte to most significant byte
  \end{itemize}
\end{frame}

% ---------------------------
\renewcommand{\currenttitle}{So what about text?}
\begin{frame}{\currenttitle}
% So we've talked a little bit about how something like an integer is stored in memory.
% But what about text?
% Everything is stored in 0's and 1's, so to store text, we'll need to basically convert text
% into numbers or sequences of numbers.
%
% The conversion process is called text encoding.
%
% Now that we've got the necessary background to talk about text encoding, we'll start looking
% at two text encoding standards, ASCII and Unicode.
  We need to convert strings of text into numbers or sequences of numbers

  \vspace*{2em}

  \tikzstyle{block} = [rectangle, draw, text width=5em,%
                       text centered, rounded corners, minimum height=4em]
  \tikzstyle{cloud} = [draw, ellipse, node distance=3cm, minimum height=2em]
  \tikzstyle{line} = [draw, -latex']
  \begin{center}
    \begin{figure}
      \begin{tikzpicture}[node distance=2cm, auto]
        \node [cloud, align=left] (input) {"Hello \\ World"};
        \node [block, right of=input, node distance=3cm] (process) {Conversion};
        \node [cloud, right of=process] (output) {\dots011101\dots};

        \path [line] (input) -- (process);
        \path [line] (process) -- (output);
      \end{tikzpicture}
      \caption{Sketch of text memory storage}
    \end{figure}
  \end{center}
\end{frame}


% -----------------------------------------------

\end{document}
